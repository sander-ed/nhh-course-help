Under market uncertainty, institutional and private investors refer to equity research analysts to provide valuable insight. These expert analysts provide buy, hold, and sell recommendations for securities with an associated target price, justifying their opinions through equity research reports. However, history has shown that even seasoned analysts struggle to beat the market and discern emerging market bubbles, as witnessed during the dot-com bubble and the global financial crisis \parencite{tuckett2009addressing}. The prevalence of biases among financial analysts is well-documented in existing literature, yet the underlying cause remains a relatively unexplored area. The relevance of studying analyst bias is heightened as more people than ever are getting into the stock market \parencite{aksjenorge2023} and there is an ongoing discussion on whether we are entering bubble territory \parencite{forbesAIbubble}. Several studies refer to the practice of extrapolating short-term growth to long-term predictions as a root cause for bias in analysts’ opinions \parencite{barberis2018extrapolation}, often referred to as trend-chasing bias. 

Previous research uses buy, hold, and sell recommendations in addition to target prices to identify analyst bias \parencite{clarkson2020target}. However, a buy, hold, or sell recommendation is a vague classification of opinions, and target prices exhibit \textit{stickiness}, reducing its utility to accurately identify analysts' current opinion on a stock \parencite{bonini2010target}. We suspect that there might be more information conveyed in their writing than observed through their recommendations. Hence, we present an alternative approach by examining the presence of trend-chasing bias on the Oslo Stock Exchange through a sentiment analysis on Norwegian equity research reports through the following research question:

\vspace{24pt}
%\vspace{50pt}
% Kanskje ha trend-chasing bias i hypotesen i stedet?
\begin{center}
    \parbox{\textwidth}{\centering\textit{Do short-term price trends impact the textual sentiment in equity research reports from Norwegian investment banks on the Oslo Stock Exchange?}}
\end{center}

% Huskeliste

    % Behavioral eller behavioural % Behavioral
    
    % Store eller små bokstaver i tabell? % Stor forbokstav (så lenge man ikke refererer til variabel)
    
    % , som Tusenskille og . som desimal? % (, som tusenskille)
    
    % Stor bokstav hver gang man refererer til Chapter, Table osv. % ja
    
    % Alltid presenter en formel, tabell o.l. % Ja
    
    % Forklar alle variabler i en formel % Husk bare de usikre, som f.eks i, t.
    
    % Kursiv vs. anførselstegn % Kursiv
    
    % Langerfeld sine kommentarer
    
    % Tabell 2.1, må jeg ha fotnote for å si at REC er Recommendation?

    % Appendix A.1 A.2 osv. 

    % Skal vi ny side for hver appendix selv om det er plass til to?

    % Ha hypotesen i innledningen lenger ned enn den står foreløpig?
    



    %Many personal and non-professional investors only see the target price and recommendation given in , making analysts tend to delay downgrades in recommendations, consequently misrepresenting analysts' true opinions at a given time. 

    %Our objective is to discern if short-term price trends affect the sentiment in equity research reports on the Oslo Stock Exchange. 









